\begin{center}
  \large\textbf{ABSTRACT}
\end{center}

\addcontentsline{toc}{chapter}{ABSTRACT}

\vspace{2ex}

\begingroup
% Menghilangkan padding
\setlength{\tabcolsep}{0pt}

\noindent
\begin{tabularx}{\textwidth}{l >{\centering}m{3em} X}
  \emph{Name}     & : & \name{}         \\

  \emph{Title}    & : & \engtatitle{}   \\

  \emph{Advisors} & : & 1. \advisor{}   \\
                  &   & 2. \coadvisor{} \\
\end{tabularx}
\endgroup

% Ubah paragraf berikut dengan abstrak dari tugas akhir dalam Bahasa Inggris
\emph{Drowsiness is a condition in which a person needs rest and often occurs unexpectedly, 
such as during studying, working, or driving. Common symptoms of drowsiness are microsleeps
(blinks longer than 500 ms) and yawning. One way to detect drowsiness is by using a non-intrusive
method using computer vision by applying the Deep Learning algorithm. RNN is a type of Deep
Learning that involves sequential input because it has memory to store state. IndRNN is a 
development of RNN but the processing of each neuron in a layer is independent of other layers.
Through this research a drowsiness classification model will be created using the parameters
of eye blink and mouth opening pattern using IndRNN. This study retrieved data in the form of
videos from the DROZY and UTA-RLDD databases. The data is then processed to obtain the Eye
Aspect Ratio (EAR) and Mouth Aspect Ratio (MAR) values for each frame. The EAR and MAR will
later be used as input for the classification model in IndRNN which produces output in the
form of 3 classes that refer to the Karolinska sleepiness scale. These classes include: First
class (Alert) which consists of class labels KSS 1 to 5, Second class (Low Vigilant) which
consists of class labels KSS 6 and 7, Third class (Drowsy) which consists of class labels
KSS 8 and 9. Based on the conducted research, the most effective drowsiness classification model 
is the IndRNN model with augmented data. It achieved an accuracy score of 0.6943 and a validation 
score of 0.52.}

% Ubah kata-kata berikut dengan kata kunci dari tugas akhir dalam Bahasa Inggris
\emph{Keywords}: \emph{Computer Vision, Drowsiness, IndRNN}.
