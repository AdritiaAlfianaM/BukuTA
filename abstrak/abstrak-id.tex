\begin{center}
  \large\textbf{ABSTRAK}
\end{center}

\addcontentsline{toc}{chapter}{ABSTRAK}

\vspace{2ex}

\begingroup
% Menghilangkan padding
\setlength{\tabcolsep}{0pt}

\noindent
\begin{tabularx}{\textwidth}{l >{\centering}m{2em} X}
  Nama Mahasiswa    & : & \name{}         \\

  Judul Tugas Akhir & : & \tatitle{}      \\

  Pembimbing        & : & 1. \advisor{}   \\
                    &   & 2. \coadvisor{} \\
\end{tabularx}
\endgroup

% Ubah paragraf berikut dengan abstrak dari tugas akhir
Pengemudi merupkan faktor utama terjadinya kecelakaan yang ada di jalanan. Hal tersebut
dilatarbelakangi oleh berbagai faktor salah satunya adalah rasa kantuk.  Gejala dari
rasa kantuk umumnya adalah microsleeps (berkedip dengan durasi lebih dari 500 ms) dan
menguap. Salah satu cara untuk mendeteksi kantuk yaitu dengan metode non-intrusif
menggunakan visi komputer dengan menerapkan algoritma \emph{Deep Learning}. \emph{Recurrent Neural Network} merupakan
salah satu jenis \emph{Deep Learning} yang melibatkan masukan berurutan karena memiliki memori
untuk menyimpan \emph{state}. IndRNN merupakan pengembangan dari RNN namun pemrosesan setiap
neuron pada suatu layer tidak bergantung dengan layer lain. Melalui penelitian ini
model klasifikasi kantuk akan dibuat dengan menggunakan parameter kedipan mata dan
pola bukaan mulut dengan menggunakan IndRNN. Penelitian ini mengambil data berupa
video dari database UTA-RLDD. Data tersebut kemudian diproses sehingga
didapatkan nilai \emph{Eye Aspect Ratio} (EAR) dan \emph{Mouth Aspect Ratio} (MAR) pada
setiap frame. EAR dan MAR tersebut nantinya akan digunakan sebagai masukan model
klasifikasi pada IndRNN yang menghasilkan output berupa 3 kelas yang mengacu pada skala
kantuk Karolinska. Kelas tersebut diantaranya : Kelas pertama (Waspada) yaitu
terdiri dari label kelas KSS 1 sampai 5, Kelas kedua (Waspada Rendah) yaitu terdiri dari
label kelas KSS 6 dan 7, Kelas ketiga (Mengantuk) yaitu terdiri dari label kelas
KSS 8 dan 9. Berdasarkan penelitian yang telah dilakukan, model klasifikasi kantuk yang 
paling efektif adalah model IndRNN dengan data teraugmentasi. Model tersebut memiliki 
nilai akurasi sebesar 0,6943 dan niali validasi sebesar 0,52.

% Ubah kata-kata berikut dengan kata kunci dari tugas akhir
Kata Kunci: IndRNN, Kantuk, Visi Komputer.
