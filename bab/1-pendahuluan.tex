\chapter{PENDAHULUAN}
\label{chap:pendahuluan}

% Ubah bagian-bagian berikut dengan isi dari pendahuluan

\section{Latar Belakang}
\label{sec:latarbelakang}

Seorang pengemudi harus bisa bertanggung jawab terhadap
keselamatan pribadinya, penumpang ataupun barang yang dibawa.
Oleh karena itu, pengemudi perlu fokus hingga sampai ke tempat
tujuan dikarenakan mengemudi ialah kegiatan yang monoton, dengan
tugas yang berulang-ulang dan memerlukan perhatian berkelanjutan
\parencite{1}. Mengantuk, didefinisikan sebagai keadaan kantuk
ketika seseorang perlu istirahat. Mengantuk dapat menyebabkan gejala yang
berdampak besar terhadap kinerja tugas seperti: waktu respons
yang melambat, kurangnya kesadaran intermiten, atau \emph{microsleeps}
(berkedip dengan durasi lebih dari 500 ms) \parencite{2}. Ciri
lain dari orang yang mengantuk adalah sering menguap pada jangka
waktu tertentu \parencite{3}. Saat mengemudi, gejala-gejala ini
sangat berbahaya karena secara signifikan meningkatkan kemungkinan
terlewatnya rambu-rambu jalan atau pintu keluar, hanyut ke jalur
lain atau bahkan kecelakaan \parencite{4}.

Mendeteksi kantuk merupakan salah satu cara untuk menghindari
dampak negatif dari rasa kantuk itu sendiri. Beberapa cara dapat
digunakan untuk mendeteksi rasa kantuk contohnya dengan metode
intrusif dimana pendeteksian rasa kantuk dilakukan dengan memanfaatkan
bantuan alat EKG (\emph{Electrocardiogram}) dan EEG (\emph{Electroencephalogram}).
Selain itu ada juga metode non-intrusif yaitu dengan melakukan ekstraksi pada
fitur wajah dengan menggunakan algoritma tertentu.

Algoritma \emph{Deep Learning} dicirikan oleh penggunaan jaringan saraf
yang dibangun dari jumlah lapisan yang besar \parencite{5}, jaringan
saraf tersebut memiliki kemampuan untuk mengotomatisasi proses ekstraksi
fitur \parencite{6}. Dalam deep learning, algoritmanya dapat mempelajari
representasi data yang lebih abstrak dan kompleks secara bertahap melalui
lapisan-lapisan yang ada dalam dalam jaringan saraf. Setiap lapisan dalam
jaringan saraf tiruan ini bertanggung jawab untuk memproses informasi secara
bertingkat, menggabungkan dan mentransformasikan data pada setiap langkahnya.

\emph{Recurrent neural networks} (RNN) merupakan salah satu jenis
dari algoritma \emph{Deep Learning} yang banyak digunakan untuk tugas
dengan input berurutan, seperti ucapan dan bahasa. Kinerja RNN lebih baik
untuk data input berurutan karena mempunyai memori yang bisa
dipergunakan untuk menyimpan state \parencite{18}. Pada tahun 2018,
Shuali Li et al. mengatakan \emph{Independently Recurrent Neural Network}
(IndRNN) yakni variasi \emph{deep learning} RNN. Bersumberkan pada
penelitian Shuali Li et al. (2018), IndRNN ini 10 kali lebih cepat
serta sedikit lebih akurat dari implementasi \emph{deep learning Long Short-Term Memory}
(LSTM) yang selalu dipergunakan \parencite{8}.

Melalui pengaplikasian IndRNN pada model klasifikasi kantuk diharapkan
adanya akurasi yang bisa terdeteksi menjadi lebih baik dari penelitian
yang sudah diselenggarakan sebelumnya. Dengan akurasi pendeteksian
kantuk yang baik maka berbagai masalah yang timbul dari rasa kantuk
saat berkendara tersebut dapat dihindari, namun tidak hanya model
yang baik, diperlukan juga sebuah sistem untuk menentukan tingkat
kantuk seseorang sehingga nantinya akan dapat digunakan sebagai bahan
penelitian yang berkaitan.

\section{Rumusan Masalah}
\label{sec:permasalahan}

Bersumberkan latar belakang pada sub-bab 1.1, penggunaan EKG dan
EEG dirasa tidak tepat untuk penggunaan sehari-hari karena memerlukan
bantuan ahli untuk menggunakan alat-alat tersebut dengan tepat.
Pengendara akan merasa kesulitan saat harus menggunakan alat-alat
penunjang sebelum mengemudikan kendaraannya. Oleh karena itu
diperlukan metode non intrusif, dimana sistem akan mendeteksi
rasa kantuk dengan mengimplementasikan \emph{deep learning} untuk
mengklasifikasikan tingkat kantuk pengemudi.

\section{Tujuan}
\label{sec:Tujuan}

Tujuan dari penelitian ini adalah membangun sistem klasifikasi
kantuk menggunakan algoritma IdnRNN dengan parameter pola kedipan
mata serta pola bukaan mulut yang dapat mengklasifikasikan kantuk
menjadi 3 kelas berbeda yaitu waspada, kewaspadaan rendah, mengantuk.

\section{Batasan Masalah}
\label{sec:batasanmasalah}

Batasan-batasan dari proyek penelitian ini adalah:

\begin{enumerate}[nolistsep]

      \item Data yang digunakan pada penelitian ini adalah video
            dari \emph{The University of Texas at Arlington Real-Life Drowsiness Dataset (UTA-RLDD)}.

      \item Model akan mendeteksi kantuk berdasarkan pola kedipan
            mata serta pola bukaan mulut.

      \item Model hanya akan memproses video masukan dengan durasi
            tercepat 9 menit dan dengan jumlah \emph{frame per second}
            sebesar 30.

      \item Model akan mengklasifikasikan video dalam 3 kelas kantuk
            yaitu kelas 1 (Waspada), kelas 2 (Kewaspadaan Rendah),
            dan kelas 3 (Mengantuk).

\end{enumerate}

\section{Manfaat}

% Ubah paragraf berikut sesuai dengan tujuan penelitian dari tugas akhir
Manfaat dari penelitian ini diantaranya :
\subsection{Bagi Penulis}
Sebagai sarana implementasi ilmu yang telah didapatkan pada masa
perkuliahan khususnya mengenai \emph{computer vision} dan \emph{deep learning}, melatih penulis untuk bisa
berpikir secara logis, sistematis, dan kritis dalam pemecahan masalah sehingga dapat
memecahkan permasalahan di kehidupan nyata seperti klasifikasi kantuk pada pengemudi.
\subsection{Bagi Institusi}
Sebagai pedoman serta inspirasi bagi penelitian lain yang
berfokus pada pemecahan masalah klasifikasi kantuk menggunakan \emph{deep learning} khususnya dengan
metode IndRNN.
\subsection{Bagi Masyarakat}
Harapannya solusi yang dibuat pada penelitian ini bisa dilaksanakan serta digunakan
menjadi produk nyata yang manfaatnya bisa dirasakan langsung oleh masyarakat sehingga dapat
mengurangi dampak negatif dari rasa kantuk itu sendiri.
