\chapter{PENDAHULUAN}
\label{chap:pendahuluan}

% Ubah bagian-bagian berikut dengan isi dari pendahuluan

\section{Latar Belakang}
\label{sec:latarbelakang}

Kantuk adalah keadaan di mana seseorang membutuhkan waktu istirahat dan seringkali
muncul pada saat yang tak terduga, seperti saat belajar, bekerja, atau mengemudi.
Mengantuk dapat menyebabkan gejala yang
berdampak besar terhadap kinerja tugas seperti: waktu respons
yang melambat, kurangnya kesadaran intermiten, atau \emph{microsleeps}
(berkedip dengan durasi lebih dari 500 ms) \parencite{2}. Ciri
lain dari orang yang mengantuk adalah sering menguap pada jangka
waktu tertentu \parencite{3}.

Salah satu masalah yang berhubungan dengan K3 (Kesehatan dan Keselamatan Kerja) yang bisa
menyebabkan kecelakaan kerja adalah kelelahan. Kelelahan kerja adalah kondisi di mana
efisiensi dan ketahanan seseorang dalam bekerja menurun, yang mengakibatkan
berkurangnya kemauan untuk bekerja. Gejala yang biasa dialami termasuk perasaan letih di
seluruh tubuh, menguap, mengantuk, kesulitan fokus dan berkonsentrasi, dan sebagainya.
Dampak dari kelelahan kerja adalah menurunnya kinerja tubuh dan produktivitas. Selain itu,
kelelahan kerja juga dapat menyebabkan timbulnya penyakit akibat kerja (PAK) dan
kecelakaan akibat pekerjaan\parencite{31}.

Mendeteksi kantuk merupakan salah satu cara untuk menghindari
dampak negatif dari rasa kantuk itu sendiri. Beberapa cara dapat
digunakan untuk mendeteksi rasa kantuk contohnya dengan metode
intrusif dimana pendeteksian rasa kantuk dilakukan dengan memanfaatkan
bantuan alat EKG (\emph{Electrocardiogram}) dan EEG (\emph{Electroencephalogram}).
Selain itu ada juga metode non-intrusif yaitu dengan melakukan ekstraksi pada
fitur wajah dengan menggunakan algoritma tertentu.

Salah satu pendekatan untuk mendeteksi kantuk adalah dengan cara intrusif, dimana fitur
HRV (heart rate variability/variabilitas detak jantung) diekstraksi untuk
merepresentasikan aktivitas sistem saraf otonom. Dalam penelitian yang dilakukan oleh
Vicente, Laguna, Bartra, dan Bailón (2016), mereka mengamati 17 pria dan 13 wanita yang
berpartisipasi dalam simulasi mengemudi. Nilai HRV diperoleh melalui penggunaan alat EEG
(Elektroensefalogram - alat untuk mengukur aktivitas listrik otak) dan EKG
(Elektrokardiogram - alat untuk mengukur aktivitas listrik jantung). Hasil penelitian
menunjukkan bahwa detak jantung cenderung melemah seiring meningkatnya rasa kantuk
\parencite{32}.

Pendekatan lain yang digunakan untuk mendeteksi kantuk adalah dengan cara non-intrusif,
yaitu dengan melakukan ekstraksi pada fitur wajah dengan menggunakan citra wajah.
Terdapat beberapa metode yang digunakan untuk menentukan area wajah diantaranya dengan
menggunakan HOG, Linear SVM, haar cascade, dan segmentasi warna. Pada penelitian ini
menggunakan metode Facial Landmark dengan versi 68-titik untuk mendeteksi area mata pada
wajah. Dalam penelitian yang dilakukan Victorio (2022), penulis membuat model deep
learning untuk mendeteksi kantuk menggunakan data uji UTA-RLDD. Untuk mengetahui apakah
pengemudi dalam kondisi mengantuk adalah mengetahui pola bukaan mulut peserta ketika
normal dan mengantuk. Hasilnya didapatkan model yang dapat mendeteksi rasa kantuk dengan
akurasi 0.87 dan loss sebesar 0.65 \parencite{10}. Terdapat penelitian pengklasifikasikan tingkat kantuk
seseorang yang  dilakukan oleh Annida (2022), penulis membuat model deep learning untuk
mengklasifikasikan kantuk menggunakan data uji DROZY. Untuk pengklasifikasian kantuk
penulis menggunakan pola bukaan mata sebagai parameternya. Hasilnya didapatkan model
yang dapat mengklasifikasikan kantuk dengan akurasi sebesar 0.976 dan loss sebesar 0.32
\parencite{11}.

\emph{Recurrent neural networks} (RNN) merupakan salah satu jenis
dari algoritma \emph{Deep Learning} yang banyak digunakan untuk tugas
dengan input berurutan, seperti ucapan dan bahasa. Kinerja RNN lebih baik
untuk data input berurutan karena mempunyai memori yang bisa
dipergunakan untuk menyimpan state \parencite{18}. Pada tahun 2018,
Shuali Li et al. mengatakan \emph{Independently Recurrent Neural Network}
(IndRNN) yakni variasi \emph{deep learning} RNN. Bersumberkan pada
penelitian Shuali Li et al. (2018), IndRNN ini 10 kali lebih cepat
serta sedikit lebih akurat dari implementasi \emph{deep learning Long Short-Term Memory}
(LSTM) yang selalu dipergunakan \parencite{8}. Selain menggunakan \emph{Deep Learning}
model klasifikasi juga bisa dilakukan menggunakan \emph{machine learning} salah satunya
adalah SVM (\emph{Support Vector Machine}). SVM menciptakan sebuah
hiperplane atau sebuah garis pemisah yang optimal dalam ruang fitur untuk
memisahkan data menjadi kelas-kelas yang berbeda. \parencite{28}

Dengan akurasi pendeteksian kantuk yang baik maka berbagai masalah yang timbul dari
rasa kantuk dihindari. Sebuah model untuk menentukan tingkat kantuk seseorang diperlukan
karena dapat digunakan sebagai bahan penelitian yang berkaitan.

\section{Rumusan Masalah}
\label{sec:permasalahan}

Bersumberkan latar belakang pada sub-bab 1.1, penggunaan EKG dan
EEG dirasa tidak tepat untuk penggunaan sehari-hari karena memerlukan
bantuan ahli untuk menggunakan alat-alat tersebut dengan tepat. Metode non
intrusif berbasis citra terbukti dapat mendeteksi kantuk tidaknya seseorang
dengan menggunakan \emph{machine learning} namun tidak banyak penelitian yang
membuat model klasifikasi kantuk dan kebanyakan hanya mengklasifikasikan kantuk
berdasarkan fitur mata saja. Oleh karena itu diperlukan model klasifikasi kantuk
berbasis citra dimana sistem akan mendeteksi rasa kantuk dengan mengimplementasikan
\emph{machine learning} untuk mengklasifikasikan tingkat kantuk pengemudi berdasarkan
fitur mata dan mulut.

\section{Tujuan}
\label{sec:Tujuan}

Tujuan dari penelitian ini adalah membangun sistem yang dapat mengklasifikasikan
tingkat kantuk  seseorang dengan parameter pola kedipan mata serta pola bukaan mulut
yang dapat mengklasifikasikan kantuk menjadi 3 kelas berbeda yaitu waspada,
kewaspadaan rendah, mengantuk \parencite{14}. Model yang diajukan adalah model klasifikasi kantuk
dengan menggunakan IndRNN.

\section{Batasan Masalah}
\label{sec:batasanmasalah}
Batasan-batasan dari proyek penelitian ini yaitu:
\begin{enumerate}[nolistsep]

      \item Data penelitian yang digunakan pada penelitian ini adalah video
            dari \emph{The University of Texas at Arlington Real-Life Drowsiness Dataset (UTA-RLDD)}.

      \item Model akan menglasifikasikan kantuk berdasarkan pola kedipan
            mata serta pola bukaan  dalam 3 kelas kantuk
            yaitu kelas 1 (Waspada), kelas 2 (Kewaspadaan Rendah),
            dan kelas 3 (Mengantuk)\parencite{14}.

      \item Model akan memproses video masukan dengan durasi
            tercepat 9 menit dan dengan jumlah \emph{frame per second}
            sebesar 30.
\end{enumerate}

\section{Manfaat}

% Ubah paragraf berikut sesuai dengan tujuan penelitian dari tugas akhir
Manfaat dari penelitian ini yaitu
\begin{enumerate}[nolistsep]
      \item Bagi Penulis

            Sebagai sarana implementasi ilmu yang telah didapatkan pada masa
            perkuliahan khususnya mengenai \emph{computer vision} dan \emph{machine learning}, melatih penulis untuk bisa
            berpikir secara logis, sistematis, dan kritis dalam pemecahan masalah sehingga dapat
            memecahkan permasalahan di kehidupan nyata seperti klasifikasi kantuk pada pengemudi.

      \item Bagi Institusi

            Sebagai pedoman serta inspirasi bagi penelitian lain yang
            berfokus pada pemecahan masalah klasifikasi kantuk menggunakan \emph{machine learning} khususnya dengan
            metode IndRNN.

      \item Bagi Masyarakat

            Harapannya solusi yang dibuat pada penelitian ini bisa dilaksanakan serta digunakan
            menjadi produk nyata yang manfaatnya bisa dirasakan langsung oleh masyarakat sehingga dapat
            mengurangi dampak negatif dari rasa kantuk itu sendiri seperti digunakan pada \emph{smart car},
            mesin industri dan yang lainnya yang mempengaruhi produktivitas individu.
\end{enumerate}