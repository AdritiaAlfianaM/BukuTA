\chapter{KESIMPULAN DAN SARAN}
\label{chap:penutup}

% Ubah bagian-bagian berikut dengan isi dari penutup
Bab ini menjelaskan mengenai kesimpulan yang telah didapat dari proses penelitian
tugas akhir.

\section{Kesimpulan}
\label{sec:kesimpulan}

Pada Tugas Akhir ini telah dijelaskan pengembangan Sistem Klasifikasi Kantuk Berdasarkan
Pola Kedipan Mata dan Pola Bukaan Mulut dengan IndRNN. Dari hasil pengujian yang telah
dilakukan pada bab sebelumnya model klasifikasi kantuk IndRNN tanpa data training yang diaugmentasi
menghasilkan akurasi sebesar 0.8378 dan akurasi validasi sebesar 0.2. Sedangkan model klasifikasi
kantuk IndRNN yang menggunakan data training teraugmentasi menghasilkan akurasi sebesar 0.6943
dan akurasi validasi sebesar 0.52, lebih baik jika dibandingkan dengan kedua model yang lainnya karena
cenderung tidak overfitting. Kedua model lain yang dibuat adalah model LSTM tanpa data training 
yang diaugmentasi menghasilkan akurasi sebesar 0.9732 dan akurasi validasi sebesar 0.32 dan model 
klasifikasi kantuk LSTM yang menggunakan data training teraugmentasi yang menghasilkan akurasi 
sebesar 0.9354 dan akurasi validasi sebesar 0.4 serta model SVM tanpa data training yang
diaugmentasi menghasilkan akurasi sebesar 0.36 pada evaluasi model dan model 
SVM yang menggunakan data training teraugmentasi menghasilkan akurasi sebesar 
0.40 pada evaluasi model.

Berdasarkan keseluruhan hasil evaluasi, didapatkan bahwa model klasifikasi
terbaik yaitu model IndRNN yang ditraining menggunakan data yang telah diaugmentasi
dengan menghasilkan 15 data benar jika dibandingkan dengan model LSTM dan SVM yang menghasilkan
10 data benar.


% \begin{enumerate}[nolistsep]
%       \item IndRNN dapat diimplementasikan untuk membuat model klasifikasi kantuk dengan
%             model LSTM dan SVM yang digunakan sebagai pembanding kinerja IndRNN.

%       \item Model klasifikasi kantuk menggunakan IndRNN tanpa data training yang
%             diaugmentasi menghasilkan akurasi sebesar 0.8378 dan akurasi validasi sebesar 0.2.
%             Sedangkan model klasifikasi kantuk IndRNN yang menggunakan data training teraugmentasi
%             menghasilkan akurasi sebesar 0.6943 dan akurasi validasi sebesar 0.52.

%       \item Model klasifikasi kantuk menggunakan LSTM tanpa data training yang
%             diaugmentasi menghasilkan akurasi sebesar 0.9732 dan akurasi validasi sebesar 0.32.
%             Sedangkan model klasifikasi kantuk LSTM yang menggunakan data training teraugmentasi
%             menghasilkan akurasi sebesar 0.9354 dan akurasi validasi sebesar 0.4.

%       \item Model klasifikasi kantuk menggunakan SVM tanpa data training yang
%             diaugmentasi menghasilkan akurasi sebesar 0.36 pada evaluasi model.
%             Sedangkan model klasifikasi kantuk IndRNN yang menggunakan data training teraugmentasi
%             menghasilkan akurasi sebesar 0.40 pada evaluasi model.

%       \item Berdasarkan keseluruhan hasil evaluasi, didapatkan bahwa model klasifikasi
%             terbaik yaitu model IndRNN yang ditraining menggunakan data yang telah diaugmentasi
%             dengan menghasilkan 15 data benar jika dibandingkan dengan model LSTM dan SVM yang menghasilkan
%             10 data benar.

% \end{enumerate}

\section{Saran}
\label{chap:saran}

Untuk pengembangan lebih lanjut mengenai Tugas Akhir ini terdapat beberapa
kemungkinan perbaikan yang dapat dilakukan untuk penyempurnaan implementasi
model antara lain:

\begin{enumerate}[nolistsep]

      \item Mencoba model deteksi wajah yang berbeda karena saat menggunakan deteksi wajah dari library dlib
            didapati beberapa titik pada wajah tidak sesuai.

      \item Melakukan metode preprocessing yang berbeda untuk menghasilkan data yang lebih baik.

      \item Mencoba menggunakan hyperparameter yang berbeda, melakukan pengaturan training yang berbeda seperti
            jumlah epoch, learning rate, serta menggunakan fungsi callback yang berbeda.

\end{enumerate}
